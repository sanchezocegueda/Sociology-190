\textbf{Question:}
In ??'s ``The Defecating Duck,'' we learn about Vaucanson's attempt to create a machine that simulates the playing of the flute.
In constructing such a machine, Vaucanson studies flute playing in great detail, and even comes up with the first physical theory of how flutes produce sound (as far as we know).
Is the main long-lasting benefit of attempting to mechanize behaviors of living beings that we gain a much better understanding of these processes?
Why do we keep hearing about automated machines replacing us, and not so much about how this process of mechanizing gives us a better understanding of the physical world and of ourselves?


\textbf{Question:}
From the Defecating Duck reading, I gather that ``imitating life'' is an ever-moving target.
The first barriers were concerned with mechanical processes (like the defecating duck).
Nowadays, it seems like the limit between humanity and artificial life lies somewhere in the space of reason, perhaps emotion and consciousness.
Assuming that these barriers will eventually be overcome, what will be the next frontier that we use to separate humanity from non-living objects such as AI?
Will we ever run out of ideas on how to differentiate ourselves from machines?

\textbf{Question:}
