This reading talks about how, with the increase in AI technology use, human interaction is become increasingly rare.
The main critique is that due to we are increasingly turning to machines to provide different services ``industrial logic.''
In particular, we are giving the task of care work to machines.
This is because the people in care work are overworked and barely have time to attend their patients, in large part due to the fact that they have to log their hours or otherwise track their statistics to measure the value of the services they provide.
Some people argue that ``AI is better than nothing,'' but the author pushes back on that claim, as AI is not free, and it comes with serious environmental repercussions.
Furthermore, what this has resulted in is that now having a human care for you is a luxury good;
only people who can afford it get to have their needs tended to by an actual human being.

\textbf{Question:}
I notice a sort of paradox here.
Automation, almost by definition, is done to make people's lives easier.
How can it be, then, that the rise of automation often comes about with more struggle, more difficult living conditions, and more stress?
Is this always the case?

