This reading talks about the automation of jobs that workers find meaningless through AI.
By meaningless, we use David Graeber's definition, where a job is meaningless if the worker deems it so.
The author gives a somewhat moot point in my opinion.
They speculate on what kinds of jobs will be automated by AI, and they also argue that this is the way that things have always been when it comes to automation.

I think it's a kind of meaningless article.
It doesn't really get to the bottom of why people are so concerned about automation in the first place.
Sure, no one wants to do meaningless jobs, but for some people, it is the requirement for you to live a comfortable life.
The fact that AI might replace your job, beyond whatever personal significance anyone can attribute to it, is concerning because it signifies that \textit{your current lifestyle is at risk}.

They only barely mention UBI towards the end, but that really only seems like a pipe dream that no one wants to implement all that seriously.
Why do we need to do ``bullshit jobs'' when AI can do them anyways?
What's the point of working so much if some tools is supposed to be able to do it better than us?
If my boring job gets replaced by AI, shouldn't I be happy?
Why can't I simply enjoy the fact that this demeaning task doesn't need to be done by any human anymore?
Why do I have to now go and find another job or risk losing everything?
We work too much.

Has this cycle of automation ever actually led to ``more valuable'' work?
Here's what I think:
large-scale automation comes about, replaces the ``meaningless'' jobs of people (typically) at the middle-to-lower socioeconomic strata, causes widespread strife and unemployment.
Meanwhile, only a handful of ``meaningful jobs'' get created, especially compared to the large amounts of people who end up unemployed.
Instead of ``meaningful jobs,'' all we get is repackaged ``meaningless jobs'' that are equally as boring as they were before.
I sincerely doubt that this automation will bring about any positive change, when it hasn't ever really done so before.
Thus, my question to the author is simple:
\textit{How?}

