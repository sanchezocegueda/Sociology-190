% This reading talks about how automation is more of an abusive ideology pushed on people in order to keep the working class afraid and insecure regarding their jobs.
% It makes many important points.
% The ones that stood out the most to me were the fact that automation is not just a technological phenomenon but an ideology.
% I also liked the point that more automation often doesn't replace work; in fact, it creates more work and rebrands it as non-work (case in point---domestic labor).
% Furthermore, there is a lot of work that is required to keep up the illusion that things are as automated as they are.

\textbf{Question:} 
%The author makes the point that we should ask ``who owns the looms?'' whenever we are trying to talk about automation.
Towards the end of the text, Astra Taylor urges us to reflect on the rhetoric regarding automation, and to not take for granted the idea that automation is an inevitable fact of life; to question who owns the means of automation and call them out for trying to 
To what extent are we doing this right now?
There seems to be a lot more consciousness about issues like this, especially online, but I am unsure if this is just an illusion that I am falling for due to the spaces I inhabit on the internet.
Is is true that our generation is more conscious about social problems, or are we just made to believe that because of the way in which information is presented online?

\textbf{Question:}
How intentional is the spread of the ``automation is inevitable'' rhetoric?
To ground my question in today's context, I see a lot of posts online made by Big Tech CEOs talking about how AI will replace everyone and their mother at whatever they do.
I get that \textit{they} spread it; they have a lot to gain from pushing these ideas into the public eye.
However, I see at least as many posts made by everyday people spreading the same myths, talking about the inevitability of the great AI replacement of the 21st century, making the exact same talking points as these CEOs.
I don't think that these people have all that much to gain from making these posts; in fact, if we agree with Taylor's text, they have a lot to lose by spreading these.
So then why do they do this?
How can we explain this phenomenon?