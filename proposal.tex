\documentclass[a4paper]{article}

\usepackage[T1]{fontenc}
\usepackage[utf8]{inputenc}
\usepackage{lmodern}

\usepackage[english]{babel}
\usepackage{csquotes}

\usepackage[notes,backend=biber]{biblatex-chicago}
\bibliography{pref}

\title{Research Proposal}
\author{Alejandro Sanchez Ocegueda}
\date{February 23, 2025}

\newcommand{\st}{$^{\text{st}}$}
\newcommand{\nd}{$^{\text{nd}}$}
\newcommand{\rd}{$^{\text{rd}}$}
\renewcommand{\th}{$^{\text{th}}$}

\begin{document}
\maketitle

% With the recent rise in popularity and proven effectiveness of Large Language Models (LLMs) in tasks involving complex reasoning, such as Dee
% the question of whether these tools are more intelligent than humans is taken 

For my final project, I want to investigate the topic of computer-assisted proofs.
``Computer-assisted proof'' is a broad term that refers to any rigorous proof of a mathematical statement that is either fully or partially completed with the aid of a computer.
Computer-assisted proofs are part of the larger field of automated reasoning, which, as the name implies, seeks to automate the process of deducing knowledge from a set of given facts.

Humans have been relying on computers (human, mechanical, or electronic) to prove theorems for centuries now.
Examples of machine-assisted proofs
Specifically, I want to attempt to answer the question: 
what process must computer assisted proofs undergo to become socially acceptable within the mathematical community?

This topic is partially compelling to me because I am working on the field of automated reasoning and machine assisted proofs myself.
The focus of my master's thesis in computer science is to develop a tool that allows users to specify and automatically verify the security of communication protocols.
Simultaneously, I am working on a project that intends to automatically formalize proofs written in natural language into proof assistants via a pipeline that includes Large Language Models (LLMs).

\end{document}