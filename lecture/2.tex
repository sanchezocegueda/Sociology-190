We started this class with a summary of the different readings.

It is important to think of the data and the different parties involved.

\section{Activity 1 - Moral Concerns}
What claims relate to automation that relate a moral concern?
For instance, in the US, we care about systemic racism.

Think of a moral concern and think about how that can be wrapped up with automation

\begin{enumerate}
    \item \textbf{Unemployment:}
    Furthering the gap of socioeconomic inequality by automating jobs that require lower levels of education.
    \item \textbf{Gender inequality:}
    Scheduling algorithms leading to weird working hours.
    You don't know when the busiest times are, and they might affect you.
    This relates to gender inequality because a lot of primary caretakers (for children) are women.
    Thus, if they are looking for jobs, they can't take the jobs because they need the regularity.
    \item \textbf{Children's development:}
    How has technology and automation affected the development of children.
    One prime example is during the COVID-19 pandemic, where children were not able to interact with other children physically.
    This relates to Gross's point that automation is permeating different parts of our lives, and how children will more likely lose out on human interaction.
    Only rich children will be able to pay for human tutors.
\end{enumerate}

\subsection{What other groups said}

\begin{enumerate}
    \item Environmental concerns: cooling systems use crazy amounts of water.
    \item Unemployment of low-paying jobs, much like the case in the zine we read.
    \item See above.
    \item Healthcare.
    ``Having telemedicine is better than no medicine.''
    But what happens when healthcare services are not as available \textit{because} of telemedicine replacing it?
    Example: if you get shot in one neighborhood and there is a hospital nearby that you can go to, you're fine.
    If you get shot in a different neighborhood where there are no emergency healthcare services, you're dead.
    \item Procrastination:
    Social media algorithms make procrastination even worse.
    Cognitive laziness -- people only use ChatGPT and do not fact-check afterwards.
\end{enumerate}


\section{Discussion on Moral Panics}
What are some criticisms to the concept of moral panics?

First of all, the term ``moral panic'' has no good definition.
It gets used so much that it loses its meaning.
In better words, they become analytically imprecise.
We use them inconsistently.
Scholars must define concepts and use them well.

The moral panic language emerges in American sociology and in British sociology.
In the UK, it was seen as a \textit{moral} panic.
In the UK it became overly political, in contrast to the US.
In the UK it was very conservative.
They were a way for sociologists to point at conservative people and call them out for being conservative and dismiss them for being alarmists and forcing their beliefs onto everyone else.

In the US, the emphasis was on the moral \textit{panic}.
The focus was on the disproportionate reactions to the different events that people were concerned about.
It was more about how the message was presented than about the message itself.

Arguments about moral panics come from a genuine place.

One focused on the moral perspective being represented and the other was focused on the way the people were talking about it.

Does sociology ever wonder why these panics arise?
Do sociologists ever wonder why people become radical in their behavior?
People are not rational, they are cultural.
Example: COVID--we were not pushing buttons, we were washing our groceries, etc.

Best says we don't need to throw out the idea of moral panic altogether.
He instead claims that moral panics are a specific type of social problem.
There are three qualities to assess whether a claim counts as a moral panic:
\begin{enumerate}
    \item \textbf{duration:} they are usually short-lived.
    \item \textbf{rhetoric:} threats to morality itself.
    They usually indicate whose morality is threatened.
    Children are often the people who are supposedly under threat by these moral panics.
    Innocents, in general.
    \item \textbf{Low-evidence:}
    They tend to be more anecdotal and ... in nature.
    It tries to scandalize people.
\end{enumerate}

\section{Human Interaction is Now a Luxury Good}
Weak ties.
Weak ties are important because they provide new sources of information.

The people who develop social media are the same people who send their kids to schools that have no-screen policies.

Grose makes empirical, qualitative data.
They shadow a person and try to extract the common experience.
Breakdown of institutional trust.

Is the only alternative ``nothing?''

Nia -- causal relation between automation and breakdown in institutions.
Capitalism?

\section{Wang \& Cruz Zine}
When we automate something, there are reverberations.
Wang \& Cruz show all the different components of the people involved in this social problem.
They show the concern from a birdseye point of view.

Owners.
Labor unions fight against automation for a long time.

\textbf{Moral entrepreneur:} wants to raise alarm bells to promote their own interests and brand.

\textbf{Moral crusades:} Don't need a critical carrying capacity for people to be outraged.
There is an A