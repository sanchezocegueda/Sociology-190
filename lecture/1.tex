% \begin{center}
% \textit{``You don't just have technology in isolation, you also have a way of using that technology.''}
% \end{center}

\textbf{January 27\th, 2025}

\section{Social problems}

There are different kinds of problems that we can consider.
The first kind of problem is a \textbf{personal problem}, which is characterized by being isolated, random, and a result from individual decision-making.
The other kind of problem that we will distinguish in this class is a \textbf{social problem}, which, in contrast to personal problems, is shared, systemic, and beyond an individual's control.

The focus of this class will be that of social problems.
These are, at their core, \textbf{socially constructed}; 
that is, they are not objective, empirical observations, but rather interpretations of common phenomena.
Social problems should not be considered a type of social conditions, but rather as a process through which these conditions are framed.

Conflict minerals.
These minerals are not intrinsically bad.
Nevertheless, the way we get these minerals can be an issue.
The extraction of minerals can be detrimental to the environment and often produce slave-like living and working conditions for the people extracting them.


\section{Noise}
There are a number of examples that we can consider as noise.
Instance, the following are all examples of things generally considered `noise:'
\begin{itemize}
    \item Construction
    \item Typing
    \item Cars
    \item Loud projectors in the classroom
\end{itemize}

Noise is a social problem, as most people agree that it is a bothersome thing.
However, noise is a rather difficult concept to define.
We may all agree that noise is a problem, but we may all differ in what exactly we \textit{mean} by noise.
What one person considers noise may not be a problem to another, and vice versa.

\section{Automation}
The phenomenon of automation---at least the version that we are experiencing right---is still very new.
We do not have a lot of sociology work on large language models, robots in food delivery.
In the sense, we in this class are on the frontline of the people who are discussing automation as a social problem.

Automation, at this point, is simply a fact of social life.
Taking Best's lead, it is best to first look at what is a fact in this world---for instance, the fact that there are robots at Kura sushi restaurants---and look at what people are \textit{saying} about these facts of life.
We are not doing the construction, but rather we are taking a look at how other people are making these claims.

\section{Discussion about the AI letter}

Central issue: 

How the issue is framed:

Actors involved:

Perspectives:

Resources:

Rhetoric:



\newpage
\section{Names of my peers}

\begin{itemize}
    \item Cheng
    \item Josh
    \item Nia
    \item Chelsea
    \item Daniel
    \item Kaelyn
    \item Elizabeth
    \item Kam
    \item Angelica
    \item Edith
    \item Zhenhan
    \item Byron
    \item Ava
    \item Chelsea K
    \item Alazne
    \item Haben
    \item 
    \item Dom
    \item Napisa
    \item Winnie
    \item Salma
    \item Jatziry F
\end{itemize}